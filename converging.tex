\section{Converging algorithms}

The algorithms we developed to the exact computation of $k$-clique communities do not scale well for huge graphs. However, starting from these algorithms, we can make them less precise but more efficient. In consequence, we are able only to have approximation of communities. However, working on an upper and on a lower approximation is very intersting if the gain of time and memory computation are important. That is what we propose here :
\begin{itemize}
\item From the algorithm which record the parasites : if we do not compute the parasites, and say that all $k$-cliques represented by the community dataset are $k$-cliques of the community, then there are $k$-cliques which appear to be in the community whereas they are not. In consequence, there are too much potential links between communities, some are merged whereas they should not. This idea gives a lower bound of communities.
\item From the algorithm which stores the communities as $(k-1)$-cliques : if we do not store all the $(k-1)$-cliques, we earn memory. But some links between current communities are broken, and at the end there are more communities than it should be. There is not enough fusion during the computation. This idea gives an upper bound of communities
\end{itemize}

More details about these ideas and its implementations are given in this section. The memory and time of computation are well improved compared to exact algorithms described precedently. Besides, the precision achieved is quite encouraging.

\subsection{Lower bound algorithm}



\subsection{Upper bound algorithm}

In this section, we give an algorithm which gives an upper bound of the number of communities from the $k$-clique percolation. This algorithm takes as a parameter the integer $z$ such as $1 \leq z \leq k-1$. The more $z$ is high and the more the algorithm is precise, but with a cost in time and memory efficiency. For $z=k-1$, the value of the bound is the exact number of $k$-clique communities.

The more we add $k$-cliques to a community, the more we increase the possibility of merging it with other communities. So, this upper bound algorithm is based on not adding all the $k$-cliques to the communities (but enough to have a relevant bound). In consequence, some communities which normally have to be merged, are not, and the bound is higher than the real number of $k$-clique communities.

Let $z$ be fixed in $\{1,...,k-1\}$.

A $z$-community is represented as :
\begin{itemize}
\item "Packs" of $k$-cliques added to the community. A pack is the set of $(k-1)$-cliques of the $k$-cliques added. Two types of packs are possible, we need to work out wich one is better :
  \begin{itemize}
  \item a big clique standing for all its $(k-1)$-sublique
  \item the exhaustive list of all $(k-1)$-subliques
  \end{itemize}

  To simplify, we first work on the exhaustive list of all $(k-1)$-subliques added.

\item List of all $z$-subcliques of the $k$-cliques added to the community 
\end{itemize}


\begin{algorithm}[!htbp]
  \caption{One pass over $k$-cliques, storing some $(k-1)$-cliques}
  \begin{algorithmic}[1]
    \State z fixed by the user : $1 \leq z \leq k-1$
    \State $UF \leftarrow$ Union-Find datastructure \Comment{For $(k-1)$-cliques}
    \State $Comz \leftarrow$ List$\{\}$ \Comment{$Comz[p]$ : Set of $z$-cliques of community $p$}
    \For {each $k$-clique $c_k \in G$}
    \State $p\leftarrow NULL$
    \For {each $(k-1)$-clique $c_{k-1}$ in $c_k$}
    \State $q\leftarrow UF.Find(c_{k-1})$
    \If {$p\neq NULL$ {\bf and} $p\neq q$}
    \State $UF.Union(p,q)$
    \State $Comz[q] = Comz[p] \cup Comz[q]$
    \EndIf
    \State $p \leftarrow q$
    \EndFor
    \State $add\_ck \leftarrow False$
    \For {each $z$-clique $c_z$ in $c_k$}
    \If {$c_z \notin Comz[p]$}
    \State $Comz[p].add(c_z)$
    \State $add\_ck \leftarrow True$
    \EndIf
    \EndFor
    \If {$add\_ck$}
    \For {each $(k-1)$-clique $c_{k-1}$ in $c_k$}
    \State $q\leftarrow UF.Find(c_{k-1})$
    \If {$q == NULL$}
    \State $q \leftarrow UF.MakeSet(c_{k-1})$
    \State $UF.Union(p,q)$
    \EndIf
    \EndFor
    \EndIf
    \EndFor
  \end{algorithmic}
\end{algorithm} 
